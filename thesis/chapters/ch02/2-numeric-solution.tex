


\section{Numerical Analysis}

In order to validate the approximations made in previous sections, we numerically solve system \ref{eq:system}. In order to do so, we use the Runge-Kutta Method of order 4 along with the  Shooting Method to solve the two point boundary value (reference to Numerical Methods in python).  



The complexity of the problem lies exactly in the two point boundary feature of the problem, where the concentrations are known in the bulk, but the potential is known at the interface(see Fig. \ref{fig:geometry}). 

The approach used in these type of problems is transforming the second order system \ref{eq:system} into a first order one,

\begin{eqnarray}
C'_+(x)-\frac{zF}{RT}E(x)C_+(x) &=& -r, \\
C'_-(x)+\frac{zF}{RT}E(x)C_-(x) &=& 0, \\
E'(x) &=& \frac{zF}{\epsilon}(C_+(x)-C_-(x)), \\
\phi'(x) &=& -E(x).
\label{eq:linear-system}
\end{eqnarray}

subject to the border conditions

\begin{eqnarray}
C_+(\delta) &= C_b,  \\
C_-(\delta) &= C_b, \\
\phi(\delta) &=& 0\\
\phi(0) &=&  V_0
\label{eq:linear-system-bc}
\end{eqnarray}

Here the primes denote derivatives with respect to $x$. Once the system is in the form of Eqn. \ref{eq:linear-system}, we can apply the Runge-Kutta Method with the shooting method to obtain the numerical solutions. 




