\section{Analytic Solution To The Diffusion-Only Problem}
\label{appendix:analytic-diff-only}
In this section we solve the diffusion problem with homogenous boundary conditions by transforming system

\begin{align}
\frac{\partial C}{\partial t} &= D \frac{\partial^2 C}{\partial x^2}
\end{align}

\begin{align}
	C(\delta, t) = -C_b, \\
	 J(0, t) &= D\frac{\partial C}{\partial x}\big|_{x=0} = 0.
\end{align}

into the dimensionless system

\begin{align}
	\rho(x,t) &= \frac{C(x,t) - C_b}{C_b}.
	\label{eq:rho-def}
\end{align}

This yields the following PDE

\begin{align}
\frac{\partial \rho}{\partial t} &= D \frac{\partial^2 \rho}{\partial x^2},\\
\label{eq:diffusion-1d}
\end{align}


The border conditions for the new function \ref{eq:rho-def} are

\begin{align}
	\rho(\delta, t) = -1, \\
	C_b D\frac{\partial \rho}{\partial x}\big|_{x=0} &= J(0, t) = 0.
\end{align}


We will solve this equation using the method of separation of variables. Let

\begin{align}
	\rho(x,t) = g(t)f(x).
\end{align}

We get the following two equations for $g(t)$ and $f(x)$


\begin{align}
	g'(t) + \lambda g(t) &= 0,\\
	f''(x) + \frac{\lambda}{D} f(x) &= 0.
\end{align}

$\lambda$ is a positive constant, due to boundary conditions. The solution to these equations are

\begin{align}
	g(t) =& g_{-}(0)e^{-\lambda t},\\
	f(x) =& F^1_{-,\lambda}\cos\qty{{\sqrt{\frac{\lambda}{D}} x}} + F^2_{-,\lambda}\sin\qty{\sqrt{\frac{\lambda}{D}} x},
\end{align}

where $F^1_{-,\lambda}$, $F^2_{-,\lambda}$ and $G_{-,0}$ are constants dependent on the parameter $\lambda$, which is yet to be determined. Border conditions \ref{eq:diffusion-1d} yield $B_{-,\lambda} = 0$. We get the general solution,


\begin{align}
	\rho(x,t) = \sum_{\lambda} A_\lambda e^{-\lambda t} \cos\qty{\frac{\lambda}{D} x},
\end{align}

where $A_{-,\lambda} = g(0)F^1_{-,\lambda}$. is a constant. Considering the border condition at $x=\delta$, we can determine the value of $\lambda$,


\begin{align}
\label{eq:general-expresion-rho}
	\rho(x=\delta,t) = \sum_{\lambda} A_\lambda e^{-\lambda t} \cos\qty{\frac{\lambda}{D} \delta} = 0.
\end{align}

The only way (unless $A_{-,\lambda} = 0$, in which case we get the trivial solution) for equation \ref{bc-delta} to be zero is if we get

\begin{align}\nonumber
	\cos\qty{\sqrt{\frac{\lambda}{D}} \delta} = 0,
\end{align}


\begin{align}
	\sqrt{\frac{\lambda}{D}} \delta = \frac{(2n+1)\pi}{2}.
\end{align}

Solving for $\lambda$,

\begin{align}
	\lambda  = \qty{\frac{(2n+1)\pi}{2\delta}}^2 D.
	\label{eq:lambda}
\end{align}

Replacing \ref{eq:lambda} into the general expression \ref{eq:general-expresion-rho},

\begin{align}
	\rho(x,t) = \sum_{n} A_n e^{-\qty{\frac{(2n+1)\pi}{2}}^2\frac{D t}{\delta^2}} \cos\qty{\frac{(2n+1)\pi}{2\delta} x}.
	\label{bc-delta}
\end{align}

From initial condition \ref{bc-delta}, we have $\rho (x,0) = -C_b $, which in turn yields
\begin{align}
	-C_b = \sum_{n}A_n\cos\qty{\frac{(2n+1)\pi}{2\delta} x}.
	\label{eq:general-expresion-rho-t0}
\end{align}

Multiplying \ref{eq:general-expresion-rho-t0} by $\cos\qty{\frac{(2m+1)\pi}{2\delta^2} x}$ and integrating over the domain, 



\begin{align}
	-C_b\int_0^\delta \cos\qty{\frac{(2m+1)\pi}{2\delta} x} dx  = \sum_{n}A_n\int_0^\delta {\cos\qty{\frac{(2m+1)\pi}{2\delta} x}\cos\qty{\frac{(2n+1)\pi}{2\delta}x} } dx.
	\label{eq:using-cos-orthogonality}
\end{align}


The integral on the LHS is simply
\begin{align}
	\int_0^\delta \cos\qty{\frac{(2m+1)\pi}{2\delta} x} dx = \qty{\frac{2\delta}{(2m+1)\pi}}\qty{\sin\qty{\frac{(2m+1)\pi}{2}}-\sin\qty{0}}.
	\label{eq:cos-integral}
\end{align}

Notice that

\begin{align}
	\sin\qty{\frac{(2m+1)\pi}{2}} = (-1)^m.
\end{align}


Therefore,

\begin{align}
	\int_0^\delta \cos\qty{\frac{(2m+1)\pi}{2\delta} x} dx = \frac{2\delta(-1)^m}{(2m+1)\pi}.
\end{align}

On the other hand, 

\begin{align}
	\int_0^\delta {\cos\qty{\frac{(2m+1)\pi}{2\delta} x}\cos\qty{\frac{(2n+1)\pi}{2\delta}x} } dx = \frac{\delta}{2}\delta_{m,n}.
\end{align}

The Fourier coefficients are thus

\begin{align}
	A_m = -\frac{4C_b}{\pi}\frac{(-1)^m}{(2m+1)}.
\end{align}


Plugging these coefficients into expression \ref{eq:general-expresion-rho-t0}

\begin{align}
	\rho(x,0) = -\frac{4C_b}{\pi} \sum_n \frac{(-1)^m}{(2m+1)}\cos\qty{\frac{(2m+1)\pi}{2\delta} x},
\end{align}

which in turn yields the concentration profile at time t=0

\begin{align}
	C(x,0) = C_b\qty{1-\frac{4}{\pi} \sum_n \frac{(-1)^m}{(2m+1)}\cos\qty{\frac{(2m+1)\pi}{2\delta} x}}.
\end{align}


From \ref{eq:general-expresion-rho-t0}, the time dependent solution for the concentration is

\begin{align}
	C(x,t) = C_b\qty{1-\frac{4}{\pi} \sum_n \frac{(-1)^m}{(2m+1)}\exp\qtys{-\qty{\frac{(2n+1)\pi}{2}}^2\frac{D t}{\delta^2}}\cos\qty{\frac{(2m+1)\pi}{2} \frac{x}{\delta}}}.
	\label{eq:solution-diffusion}
\end{align}


We define the dimensionless parameters $\tau = \frac{D t}{\delta^2}$ and $\xi = \frac{x}{\delta}$. In terms of this parameters the concentration is
\begin{align}
\label{eq:solution-diffusion-appendix}
	C(\xi,\tau) = C_b\qty{1-\frac{4}{\pi} \sum_n \frac{(-1)^m}{(2m+1)}\exp\qtys{-\qty{\frac{(2n+1)\pi}{2}}^2\tau}\cos\qty{\frac{(2m+1)\pi}{2} \xi}}.
\end{align}

