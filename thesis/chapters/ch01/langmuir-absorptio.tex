\section{Langmuir Adsorption Model}

Adsorption is a process of adding a particular substance to the surfaces of a solid. This is different than absorption where the substance is actually absorbed into the solid.

Consider the following scenario: we have a substance $B_{(l)}$ in liquid state and $B_{(s)}$ in solid state. The total amount of $B$ substance is

\begin{align}
	VC_B = QC_{(l)} + S C_{(s)}.
\end{align}

Since the interaction can only be at the surface, then $V = Q$. Thus,

\begin{align}
	C_{(s)} = \frac{QC_B - QC_{(l)}}{S}.
\end{align}

Which yields the amount of substance that is adsorbed to the surface. 

Adsorption depends on the amount of available sites for substance $B$ to attach itself into. The general equation for the reaction is

\begin{align}
	C_{(l)} + \left[\text{Open Sites}\right] \rightleftharpoons C_{(s)}.
\end{align} 

Therefore the equilibrium constant for this reaction is

\begin{align}
	K_{eq} = \frac{[C_{(s)}]}{[C_{(l)}][\text{Open Sites}]}.
\end{align}

Let $S_T$ be the total number of sites available for this process. The number of total available sites at a given point will be $S_T -C_{(s)}$. Therefore

\begin{align}
	K_{eq} = \frac{[C_{(s)}]}{[C_{(l)}][S_T -C_{(s)}]}.
\end{align}

Solving for $C_{(s)}$

\begin{align}
	[C_{(s)}] = \frac{K_{eq}[S_T][C_{(l)}]}{1+K_{eq}[C_{(l)}]}.
\end{align} 

To calculate the equilibrium constant of the reaction, we can use the Gibb's free energy, which is

\begin{align}
	\Delta G^0 = -RT \log\qty{K_{eq}}, \\
	\Delta G^0 = -zF V_0,
\end{align}

where $V_0$ is the electrolytic cell's potential and $z$ is the electrolyte's valence. The equilibrium constant can be found to be

\begin{align}
	K_{eq} = \exp\qty{\frac{zF}{RT}V_0}.
\end{align}

Therefore, by knowing the concentration of sites of attachment for $B$ we obtain the adsorption model known as the Langmuir isotherm (it is called an isotherm because the equilibrium constant also depend on temperature) \cite{langmuir}.





