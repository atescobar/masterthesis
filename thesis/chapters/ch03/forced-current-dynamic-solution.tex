\section{Forced Current Setup}

In the previous section we used a concentration dependent boundary condition. That means that the current is a consequence of the chemical reaction at the interface. In this section we would like force the current. The flux at the interface is defined as

\begin{align}
	\label{eq:forced-current-boundary}
    J_+(x = 0) &= -\mathcal{D}_+\qty{\frac{\partial C_+}{\partial x} - C_+ \frac{z\mathcal{F}}{RT}\frac{\partial\phi}{\partial x}}\bigg|_{x= 0}= R,\\
    J_-(x = 0) &= -\mathcal{D}_-\qty{\frac{\partial C_-}{\partial x} + C_- \frac{z\mathcal{F}}{RT}\nabla\phi} \bigg|_{x= 0} = 0,\\
\end{align}

since only the positive electrolyte reacts with the electrode. The chemical reaction can be written in terms of the concentration as

\begin{align}
	R = -k_f C(t, x=0) = \frac{i_0}{\mathcal{F}},
\end{align}

where $k_f$ is the ......, $i_0$ is the current density (current by unit area) and $\mathcal{F}=N_A e$ the Faraday constant.

Discretizing boundary condition \ref{eq:forced-current-boundary} and letting $C_s = C_b\rho_s$ we get

\begin{align}
	\rho_+^{n,k}=\frac{\rho_+^{n,k+1}-R\Delta \xi/\mathcal{D_+}Cb}{1+\qty{\psi^{n,k+1}-\psi^{n,k}}},\\
	\rho_-^{n,k}=\frac{\rho_-^{n,k+1}}{1-\qty{\psi^{n,k+1}-\psi^{n,k}}},
\end{align}

surfaceDeltaE

forced-current-nernst